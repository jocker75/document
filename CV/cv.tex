\documentclass{Ghanem_cv}

\setname{Abderrahmane}{GHANEM}
\setaddress{Paris/France}
\setmobile{06 65 59 98 58}
\setmail{ramane.ghanem@gmail.com}
\setposition{Work Student} 
\setlinkedinaccount{https://www.linkedin.com/in/abderrahmane-ghanem-a4673b191/}
\setgithubaccount{https://github.com/jocker75/document/} 
\setthemecolor{red}

\begin{document}
\headerview
\vspace{1ex}
%
%Présentation
\addblocktext{Présentation}{
Passionné par les statistiques, le machine learning, l'apprentissage profond et doté d'une forte capacité d'adaptation et possédant un bon relationnel, je suis à la recherche d'un premier emploi afin que je puisse évoluer et m'intégrer dans l'environnement de l'entreprise.
}
%
%Formation
\section{Formation} 
    \datedexperience{Université Pierre et Marie Curie}{2018-2020} 
    \explanationdetail{\coloredbullet\ 
    Master Statistique\par} 
    \datedexperience{Université Pierre et Marie Curie}{2015-2018} 
    \explanationdetail{\coloredbullet\ Licence Mathématique Appliquée\par} 
%
% Expériences Académique
\section{Expériences Académique}
    %
    \datedexperience{Stage de fin d'étude.}{2020/ France} 
    \explanation{LABORATOIRE IMAGES, SIGNAUX ET SYSTÈMES INTELLIGENTS} 
    \explanationdetail{\coloredbullet\ Lecteur de différents articles scientifiques.
    }
    \explanationdetail{\coloredbullet\ Réalisation d'un modèle à plusieurs couches cachées.
    }
    \explanation{Environnements de développement : Google Colaboratory \cpshalf Langage : Python} 
    %
     \datedexperience{Monte Carlo séquentielles}{2019-2020 / Master 2} 
    \explanationdetail{\coloredbullet\ %
     Implémentation de la méthode Monté Carlo séquentielle sous R.
     }
    \explanationdetail{\coloredbullet\ %
     Réalisation de différentes simulation.
     }
    %
    \datedexperience{K-plus proches voisins}{2018-2019 / Master 1} 
    \explanationdetail{\coloredbullet\ %
     Implémentions de la méthode des k plus proches voisins.
     }
    \explanationdetail{\coloredbullet\ %
     Comparaison entre la méthode implanté et la méthode de sklearn.
     }
    \explanation{ENVIRONNEMENTS DE DÉVELOPPEMENT : Rstudio , Rshiny \cpshalf Langage : R} 
    %
    \datedexperience{Kaggle}{Depuis 2019} 
    \explanation{Je participe au différents compétitions proposé sur le site ens, pour augmenter mes compétences en machine learning} 
    %
    
%Autre Expérience
\section{Autre Expériences }
    %
    \datedexperience{Cours particulier en mathématique}{2017-2019 / Clamart, France} 
    \explanationdetail{\coloredbullet\
    Aide aux devoirs en général, particulièrement en mathématique \par} 
    %
    \datedexperience{Vendeur au comptoir de produits frais}{Depuis 2017 / Monoprix Meudon, France} 
    \explanationdetail{\coloredbullet\
    Travailler en équipe, suivre les instructions de mes supérieurs et être à l'écoute du client\par} 
    
    
%Compétences    
\section{Compétences}
    %
    \newcommand{\skillone}{\createskill{Programmation}{\textbf{\emph{Bon niveau:}} \ \  Python \cpshalf R \ \ \textbf{\emph{débutant:}} \ \  Matlab \cpshalf SAS }}
    %
    \newcommand{\skilltwo}{\createskill{Environnements de
Développement}{Jupyter \cpshalf RStudio \cpshalf Rshiny \cpshalf Spyder \cpshalf Notebook\cpshalf GIT}}
    %
    \newcommand{\skillthree}{\createskill{ Bibliothèques}{ Matplotplib \cpshalf Numpy \cpshalf Pandas \cpshalf Scikit-learn \cpshalf Gym \cpshalf PyTorch \cpshalf Tensorflow}}
    %
    \newcommand{\skillfive}{\createskill{Langues}{\textbf{\emph{Bon niveau:}} \ \  Anglais \ \ \textbf{\emph{Courant:}} \ \ Français \ \ \textbf{\emph{Courant:}} \ \  Arabe }}
    %    
    \newcommand{\skillfour}{\createskill{Statistique}{Analyse qualitative \& quantitative \cpshalf Tests Statistiques \cpshalf Estimations statistiques }}
    %
    \createskills{\skillone, \skilltwo, \skillthree, \skillfour, \skillfive}

\end{document}


